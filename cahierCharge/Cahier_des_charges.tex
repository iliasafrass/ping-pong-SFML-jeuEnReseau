 \documentclass[a4paper,oneside]{article}
 \usepackage[frenchb]{babel}
 \usepackage[utf8]{inputenc}
 \usepackage[T1]{fontenc}
 \usepackage{graphicx}
 \usepackage{amssymb} 
 \usepackage{amsmath}
 \usepackage{hyperref}
 \usepackage{fullpage}
 \usepackage{epstopdf}
 %%%%%%%%%%%%%%%%%%%%%%%%%
 \newcommand{\mytitle}{Projet Pong - Cahier des charges}
 \title{\mytitle } 
 %%%%%%%%%%%%%%%%%%%%%%%%%
 \makeatletter
 \usepackage{fancyhdr}
 \pagestyle{fancyplain}
 \fancyhf{}
 \renewcommand{\headrulewidth}{0pt}
 \renewcommand{\footrulewidth}{0.5pt}
 \lfoot{\mytitle}
 \cfoot{\@date}
 \rfoot{page \thepage / \pageref{fin}}
 \author{L3 Informatique }
 \date{6 juin 2015}
 %%%%%%%%%%%%%%%%%%%%%%%%%
 \begin{document}
 	\maketitle
 	\thispagestyle{fancyplain}
 	%%%%%%%%%%%%%%%%%%%%%%%%%
 	\section{Renseignements}
 	\paragraph{Nom du projet :}
 	Pong
 	\paragraph{Objet :}
 	Développement d'un jeu en réseaux nommé \og Pong \fg{}
 	\paragraph{Maître d'ouvrage :}
 	Club Ping-PONG
 	\paragraph{Maître d'oeuvre : }
 	Taibi Anas\footnote{Contact: taibianas1994@gmail.com}, Afrass Ilias\footnote{Contact : iliasafrass@gmail.com}, Adansar Mohamed\footnote{Contact : a1h@hotmail.fr} et Abakarim Marouan\footnote{Contact : mar.abakarim@gmail.com}
 	\paragraph{Date de début :}
 	6 juin 2015
 	\paragraph{Date de fin :}
 	22 juin 2015
 	%%%%%%%%%%%%%%%%%%%%%%%%%
 	\newpage
 	\section{Définition du besoin}
 	\paragraph{Contexte général\\}
	  Un club de ping-pong souhaite la réalisation d'un jeu en temps réel représentant une partie de ping-pong.Notre travail consiste à concevoir ce jeu tout en respectant leurs besoins.
 
 	\paragraph{Besoins et priorités\\}
 	Le besoin principal est de pouvoir jouer le jeu en réseaux, contrôler la barre, le mouvement de la balle, calculer le score et de déterminer le vainqueur.
 	L'interface doit être simple et efficace.
 	
 	\paragraph{Les règles principales du jeu}
	 	\begin{enumerate}
	 		\item La partie se joue seulement en deux joueurs qui seront liés en	réseaux.
	 		\item Une balle se déplace et se rebondit sur les bords	
	 		supérieure et inférieure de la fenêtre.
	 		\item Il aura deux barres qui seront déplaçables verticalement par	les joueurs.
	 		\item Lorsqu'un joueur (barre) touche la balle, cette	
	 		dernière change de direction et accélère (augmente	
	 		la vitesse).
	 		\item Lorsque la balle touche un côté (les côtés derrière	
	 		les barres) de la fenêtre, la balle revient au milieu de nouveau	et on recommence le jeu tout en ajoutant un point au score du joueur adverse. 
	 		\item La partie se finit lorsque le score d'un joueur arrive
	 		à 10 ou l'un des joueurs quitte la partie.
	 		\end{enumerate}
	 		\paragraph{Fonctionnalités:}
			\begin{enumerate}
				\item Du serveur
				\begin{itemize}
					\item Lancer Serveur.
					\item Créer des parties entre deux clients.
					\item Lancer partie.
					\item Gestion de la collision.
					\item afficher le gagnant.
				\end{itemize} 	
				\item{Du client}
				\begin{itemize}
					\item Exécuter le programme.
					\item Afficher le jeu.
					\item Jouer (faire bouger les barres).
				\end{itemize}
					
			\end{enumerate}

 	%%%%%%%%%%%%%%%%%%%%%%%%%
 	\newpage
 	\section{Description Technique Du Projet}
 	\begin{enumerate}
 	\item	\textbf{ \underline{Le but et le principe de fonctionnement de l'application:}}
 		
 		
	 		Le but principal de l'application est de présenter le sport avec un jeu en
	 		
	 		remplaçant les deux joueurs par deux barres, La balle partira du milieu de l'écran avec
	 		
	 		un déplacement initial, lorsqu'elle touchera une partie de l'écran ou une
	 		
	 		barre , elle rebondira et ira dans l'autre sens.
	 		
		 \item \textbf{\underline{Les principaux objets :}}
		 \begin{description}
		 	\item[La barre : ]l'objet qui représente les joueurs, et qui sert à frapper la balle.
		 	\item[La balle : ]l'objet qui se déplace entre les deux barres.
		 	\item[La fenêtre du jeu :] elle arrange tous les objets de notre jeu.
		 \end{description}
	 		
 	\end{enumerate}
 	\section{Spécifications}
 	\begin{enumerate}
 		
  		\item Les joueurs peuvent contrôler le jeu soit par clavier ou par	souris.
  		\item Celui qui arrive à 10points le premier est le	gagnant.
  		\item interface graphique pour écran tactile de résolution $800 \times 600$
 		\item affichage simple avec zones tactiles d'au moins $100 \times 100$ pixels
 	
 	\end{enumerate}
 	%%%%%%%%%%%%%%%%%%%%%%%%%
 	\newpage
 	\appendix
 	\section{Livrables.}
 	\begin{itemize}
 		\item logiciel programmé en \href{https://fr.wikipedia.org/wiki/C\%2B\%2B}{C++} ET \href{http://www.sfml-dev.org/index-fr.php}{SFML} 
 		\item code source documenté sous licence \href{http://www.gnu.org/licenses/gpl-3.0.fr.html}{GNU gpl-3.0} 
 		\item manuel d'installation et de configuration
 		\item manuel d'utilisation
 	\end{itemize}
 	\section{Diagrammes de classe}
 	\begin{itemize}
 		\item Partie client.
 		\end{itemize}
 		\includegraphics[width=1.1\linewidth,height = 0.4\textheight]{Diagramme_classe_client.png}
 		
	 ~\\
 		 	\begin{itemize}
 		 		\item Partie serveur.
 		 	\end{itemize}
 		 	\includegraphics[width=1.1\linewidth, height = 0.3\textheight]{Diagramme_classe_serveur.png}
 		 		
 	
 	%\includegraphics[width=11cm]{}
 	\section{Maquettes}
 	
 	\paragraph{Page utilisation}
 	~\\
 	\includegraphics[width=15cm]{Capture_ecran_page_de_jeu.png}
 	~\\
 
 	\begin{center}
 		avec 1 comme score du  premier joueur et 0 du deuxième.
 		
 	\end{center} 	
 	
 	%%%%%%%%%%%%%%%%%%%%%%%%%
 	\label{fin}
 \end{document}